\documentclass{article}\n\usepackage{amsmath}\usepackage{graphicx}\usepackage{hyperref}\usepackage{fancyhdr}\n\usepackage{geometry}\n\geometry{a4paper, margin=1in}\n\newcommand{\custombox}[1]{\begin{center}\fbox{\begin{minipage}{0.9\linewidth}#1\end{minipage}}\end{center}}\n\begin{document}\n\title{Environmental Studies AEC 01}\n\author{Department of Environmental Science}\n\date{\today}\n\maketitle\n\tableofcontents\n\newpage\n\section{Course Outcomes}\nUpon completion of this course, students will be able to: \n\begin{itemize}\n  \item Understand the concepts of abiotic and biotic components. \n  \item Analyze biodiversity and its importance. \n  \item Assess the biogeochemical cycles and their implications. \n  \item Evaluate energy flow in ecosystems. \n  \end{itemize}\n\newpage\n\section{Unit 1: Ecosystem Dynamics}\n\subsection{Abiotic and Biotic Components}\nEcosystems consist of both abiotic factors (like temperature, water, and minerals) and biotic factors (like plants and animals) that interact with each other.\n\subsection{Biodiversity}\nBiodiversity refers to the variety of life in a particular habitat or ecosystem. It is crucial for ecosystem resilience.\n\subsection{Biogeochemical Cycles}\nThese cycles describe the movement of elements through biological and physical systems.\n\newpage\n\section{Unit 2: Pollution Studies}\n\subsection{Air Pollution}\nAir pollution is the contamination of the atmosphere and it affects climate and health.\n\subsection{Water Pollution}\nWater pollution causes ecosystem degradation and poses health risks to humans and wildlife.\n\newpage\n\section{Unit 3: Soil and Water Analysis}\n\subsection{Methods for Soil Analysis}\nTechniques include pH testing, nutrient content analysis, and moisture determination.\n\subsection{Water Quality Testing}\nMethods involve chemical tests, biological indicators, and physical assessments of water bodies.\n\newpage\n\section{Unit 4: Remediation Techniques}\n\subsection{Microbes for Pollution Control}\nBioremediation uses microorganisms to degrade environmental contaminants.\n\subsection{Biodegradation and Phytoremediation}\nThese processes leverage biological organisms to decompose pollutants and utilize plants for cleaning environments.\n\newpage\n\section{Glossary}\n\begin{itemize}\n  \item Biotic - Living components of an ecosystem.\n  \item Abiotic - Non-living chemical and physical factors in the environment.\n  \item Biodiversity - Variety of life in the world or a particular habitat.\n  \item Biogeochemical Cycle - Pathways by which elements and compounds move through the Earth system.\n  \item\ldots (40+ more terms)\n\end{itemize}\n\newpage\n\section{References}\n\begin{itemize}\n  \item Reference 1\n  \item Reference 2\n  \item Reference 3\n  \end{itemize}\n\newpage\n\end{document}